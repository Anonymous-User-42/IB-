\documentclass[12pt, a4paper]{article}
\usepackage[utf8]{inputenc}
\usepackage{parskip}
\usepackage{amsmath}
\usepackage{amssymb}
\usepackage{float}
\restylefloat{table}
\usepackage{diagbox}
\usepackage{bm}
\usepackage{import}
\usepackage{hyperref}
\usepackage{titlesec}
\usepackage{lipsum}
\usepackage{graphicx}
\usepackage{tikz}
\usepackage[export]{adjustbox}
\usepackage{etoolbox}
\usepackage{listings}
\usepackage{xcolor}
\lstloadlanguages{Python}
\usepackage{textgreek}
\usepackage{wrapfig}
\usepackage{pgfplots}
\pgfplotsset{width=14cm,compat=1.17}
\usepgfplotslibrary{external}
\usetikzlibrary{patterns.meta,decorations.pathmorphing,patterns}
\tikzexternalize
\usepackage[]{babel}
\usepackage{appendix}
\usepackage{listingsutf8}
%\usepackage[scaled,nueu]{helvet}
%\usepackage[proportional,scaled=1.064]{erewhon}
%\usepackage[T1]{fontenc}
\usepackage{booktabs}
\usepackage{multirow}
\usepackage{mathptmx}



\definecolor{codegreen}{rgb}{0,0.6,0}
\definecolor{codegray}{rgb}{0.5,0.5,0.5}
\definecolor{codepurple}{rgb}{0.58,0,0.82}
\definecolor{backcolour}{rgb}{0.95,0.95,0.85}

\lstdefinestyle{mystyle}{
    backgroundcolor=\color{backcolour},   
    commentstyle=\color{codegreen},
    keywordstyle=\color{magenta},
    numberstyle=\tiny\color{codegray},
    stringstyle=\color{codepurple},
    basicstyle=\ttfamily\footnotesize,
    breakatwhitespace=false,         
    breaklines=true,                 
    captionpos=b,                    
    keepspaces=true,                 
    numbers=left,                    
    numbersep=5pt,                  
    showspaces=false,                
    showstringspaces=false,
    showtabs=false,                  
    tabsize=2
}

\lstset{style=mystyle}

\lstset{literate={θ}{{$\theta$}}1{ω}{{$\omega$}}1}




%\renewcommand{\familydefault}{\sfdefault}

\renewcommand{\contentsname}{\textit{Contents}\hline}
%\renewcommand{\refname}{\textit{Bibliography}}
%\renewcommand{\refname}{{Bibliography}}
\renewcommand{\refname}{{Works Cited}}
\renewcommand{\listfigurename}{\textit{List of Figures}\hline}
\renewcommand{\listtablename}{\textit{List of Tables}\hline}

\nocite{*}

\titleformat{\chapter}{\huge\bfseries}{\thechapter}{0.5em}{\huge}
\makeatletter
\patchcmd{\chapter}{\if@openright\cleardoublepage\else\clearpage\fi}{}{}{}
\makeatother



\graphicspath{{Images/}}





\begin{titlepage}
    \begin{center}
        \vspace*{1cm}
            
        % False Date Parameter
        \date{}
            
        \huge
            
        \textit{\textbf{Physics HL \\ Internal Assessment}}
            
        \vspace{0.25cm}
            
%        \hline
            
        \vspace{2.5cm}
            
%        \textit{\textbf{The Fluid Dynamics of a Spherical Object}}
            
%		\vspace{1.5cm}            
            
        \vspace{2.5cm}
            
        \LARGE
            
        \textit{\textbf{How does Temperature of the fluid medium in laminar flow affect the drag force on a spherical body in linear motion? }}

%		False Parameter deletion requested if only the RQ parameter is to be deployed

		\vspace{1.5cm}
            
        \vspace{2.5cm}
            
        \Large
            
        \vspace{0.25cm}
        
%		\textit{\textbf{Word Count}: 3990}            

		\vspace{1cm}            
            
		\Large		
		        
		\vspace{0.25cm} 
		
%		\textit{\textbf{Word Count}: 2860}		

%		\textit{\textbf{Number of Pages}: }

		\textit{\textbf{Page Count}: 19}
		   
%        \textit{\textbf{Mohammed Sayeed Ahamad}}
            
        \vspace{2cm}
            
        \Large
         
		\vspace{0.25cm}         
            
%        \textit{\textbf{Batch of 2022}}           
            
        \vspace{0.25cm}
            
%        \hline
           
            
    \end{center}
\end{titlepage}







\begin{document}

\maketitle

\tableofcontents
\clearpage
%\listoffigures
%\listoftables



%\begin{abstract}
%    \import{Sections}{Abstract}
%\end{abstract}



%\large
			
%		{\textbf{Research Question}: \textit{How does Temperature of the fluid medium in laminar flow affect the drag force on a spherical body in linear motion?} }
			
%\normalsize

\section{{Research Question}}
		
	\import{Sections}{Research Question}
		
\section{{Introduction}}
        
        \import{Sections}{Introduction}       
        
\section{{Aim}}
        
        \import{Sections}{Aim}

%	Research Question
			
%\large
			
%		{\textbf{Research Question}: \textit{How does Temperature of the fluid medium in laminar flow affect the drag force on a spherical body in linear motion?} }
			
%\normalsize
				
\section{{Hypothesis}}
        
        \import{Sections}{Hypothesis}    
        
\section{{Background Research}}
        
        \import{Sections}{Background Research}
        
\section{{Materials Required}}
        
        \import{Sections}{Materials Required}
        
\section{{Variables}}
        
        \import{Sections}{Variables}
        
\section{{Procedure}}
        
        \import{Sections}{Procedure}

%		\import{Sections}{Procedure Mod 1}
        
%\chapter{\textit{Initial Conditions}}
        
%        \import{Sections}{Initial Conditions}
        
\section{{Equations}}
        
        \import{Sections}{Equation}

\section{{Data}}

	  \import{Sections}{Data}      
        
%\chapter{\textit{Experimental Data}}
        
%        \import{Sections}{Experimental Data Mod 2}
        
%\chapter{\textit{Simulation Data}}
        
%        \import{Sections}{Simulation Data}
        
%\chapter{\textit{Derived Data}}
        
%        \import{Sections}{Derived Data}
        
%\section{\textit{Observations}}
        
%        \import{Sections}{Observations}
        
%\chapter{\textit{Observations from Derived Data}}
        
%        \import{Sections}{Observations from Derived Data}

\section{{Evaluation}}

	\import{Sections}{Evaluation Mod 2}
        
\section{{Analysis}}
        
%        \import{Sections}{Graphical Analysis}

			\import{Sections}{IAGraph}
        
%        \import{Sections}{Numerical Analysis}
        
%\section{\textit{Evaluation}}
        
%        \import{Sections}{Evaluation}
        
\section{{Limitations of Study}}
        
        \import{Sections}{Limitations of Study}
    
\section{{Safety Measures}}
        
        \import{Sections}{Safety Measures} 
                
\section{{Sources of Error}}
        
        \import{Sections}{Sources of Error}
        
\section{{Conclusion}}
        
        \import{Sections}{Conclusion}
        

        
\clearpage

\import{Works Cited}{WC}
\import{Works Cited}{Bib v2}


\clearpage

\appendix

\appendixpage

\import{Appendix}{appdix Mod 1}


\end{document}

\let\clearpage\relax

\makeatletter
\patchcmd{\chapter}{\if@openright\cleardoublepage\else\clearpage\fi}{}{}{}
\makeatother

\lstset{literate={ω}{{$\omega$}}1}



aim
objectives
hypothesis
rationale and purpose of study -> anticipated result's of this study
Testing Hypothesis
literature review
input and output based research
results - conclusion

methodology -> Data source
Data collection methodology - published literature
method of data analysis

Dissertation topic

Recommendations 


\section{\textit{Sub case when m = 100 g}}
        
\section{\textit{Sub case when m = 200 g}}
        
\section{\textit{Sub case when m = 300 g}}
        
\section{\textit{Sub case when m = 400 g}}
        
\section{\textit{Sub case when m = 500 g}}
        
\section{\textit{Sub case when m = 600 g}}
        
\section{\textit{Sub case when m = 700 g}}
        
\section{\textit{Sub case when m = 800 g}}
        
\section{\textit{Sub case when m = 900 g}}
        
\section{\textit{Sub case when m = 1000 g}}
        





\begin{table}[H]
                \centering
                \begin{tabular}{|c|c|c|c|c|c|c|c|c|c|}
                \hline
                \hline
                \diagbox[width=12em]{\textit{Observation} \\ \textit{number} ($x_i$)}{\textit{Time} \\ (\textit{in seconds})} & 0 & 2.5 & 5.0 & 7.5 & 10.0 & 12.5 & 15.0 & 17.5 & 20.0 \\
                \hline
                \hline
                1 &  &  &  &  &  &  &  &  &  \\
                \hline
                2 &  &  &  &  &  &  &  &  &  \\
                \hline
                3 &  &  &  &  &  &  &  &  &  \\
                \hline
                4 &  &  &  &  &  &  &  &  &  \\
                \hline
                5 &  &  &  &  &  &  &  &  &  \\
                \hline
                6 &  &  &  &  &  &  &  &  &  \\
                \hline
                7 &  &  &  &  &  &  &  &  &  \\
                \hline
                8 &  &  &  &  &  &  &  &  &  \\
                \hline
                9 &  &  &  &  &  &  &  &  &  \\
                \hline
                10 &  &  &  &  &  &  &  &  &  \\
                \hline
                \hline
                \end{tabular}
                \caption{}
                \label{}
            \end{table}
            
            
            
            
            
            
            


φ





