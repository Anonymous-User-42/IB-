{Let $F_{{D}_{Exp}}$ be the experimental values and $F_{{D}_{Sim}}$ be the simulation values respectively for each and every case that we are investigating.} 

{If we define $F_{D_n}$ as the uncertainty in measurement in the experimental values of $F_D$, then $F_{D_n}$ is mathematically defined as $\left| F_{{D}_{Sim}} - F_{{D}_{Exp}} \right|$}

{By using the above definitions we have,}

\subsection{{Observation from Case 1 where $T = 10^\circ$C}}

	{By using equation \ref{eq1}, we see that the experimental and simulation value for $F_{D_1}$ is 4.6 mN and 4.6 mN.}
        
	{Therefore uncertainty in measurement for $F_{D_1}$ in this case is $\pm{\textit{0}}$ N.}        
        
\subsection{{Observation from Case 2 where $T = 20^\circ$C}}

	{By using equation \ref{eq1}, we see that the experimental and simulation value for $F_{D_2}$ is 4.5906 mN and 4.59 mN.}
        
	{Therefore uncertainty in measurement for $F_{D_2}$ in this case is $\pm{\textit{0.0006}}$ mN.}        
        
\subsection{{Observation from Case 3 where $T = 30^\circ$C}}

	{By using equation \ref{eq1}, we see that the experimental and simulation value for $F_{D_3}$ is 4.631 mN and 4.56 mN.}
        
	{Therefore uncertainty in measurement for $F_{D_3}$ in this case is $\pm{\textit{0.071}}$ mN.}        
        
\subsection{{Observation from Case 4 where $T = 40^\circ$C}}
        
	{By using equation \ref{eq1}, we see that the experimental and simulation value for $F_{D_4}$ is 4.527 mN and 4.53 mN.}
        
	{Therefore uncertainty in measurement for $F_{D_4}$ in this case is $\pm{\textit{0.297}}$ mN.}       
        
\subsection{{Observation from Case 5 where $T = 50^\circ$C}}

	{By using equation \ref{eq1}, we see that the experimental and simulation value for $F_{D_5}$ is 4.503 mN and 4.5 mN.}
        
	{Therefore uncertainty in measurement for $F_{D_5}$ in this case is $\pm{\textit{0.03}}$ mN.}        
        
\subsection{{Observation from Case 6 where $T = 60^\circ$C}}
        
	{By using equation \ref{eq1}, we see that the experimental and simulation value for $F_{D_6}$ is 4.4606 mN and 4.46 mN.}
        
	{Therefore uncertainty in measurement for $F_{D_6}$ in this case is $\pm{\textit{0.0006}}$ mN.}
        
\subsection{{Observation from Case 7 where $T = 70^\circ$C}}
        
	{By using equation \ref{eq1}, we see that the experimental and simulation value for $F_{D_7}$ is 4.4246 mN and 4.42 mN.}
        
	{Therefore uncertainty in measurement for $F_{D_7}$ in this case is $\pm{\textit{0.0046}}$ mN.}
        
\subsection{{Observation from Case 8 where $T = 80^\circ$C}}

	{By using equation \ref{eq1}, we see that the experimental and simulation value for $F_{D_8}$ is 4.3666 mN and 4.37 mN.}
        
	{Therefore uncertainty in measurement for $F_{D_8}$ in this case is $\pm{\textit{0.0534}}$ mN.}        
                
\subsection{{Observation from Case 9 where $T = 90^\circ$C}}
        
	{By using equation \ref{eq1}, we see that the experimental and simulation value for $F_{D_9}$ is 4.3166 mN and 4.32 mN.}
        
	{Therefore uncertainty in measurement for $F_{D_9}$ in this case is $\pm{\textit{0.0034}}$ mN.}
        

        

        




{We further define, the average uncertainty in measurement across all cases that have been investigated to be,} 

    $$\overline{F_{D_n}} = \frac{\sum_{n=1}^{n}F_{D_n}}{n} = \frac{F_{D_1} + F_{D_2} + F_{D_3} + F_{D_4} + F_{D_5} + F_{D_6} + F_{D_7} + F_{D_8} + F_{D_9}}{9}$$

{Therefore we have,}

    $\overline{F_{D_n}} = 0.05112 mN = 5.112\times 10^{-6} N$

{Upon observation, we see that the value of $\overline{F_{D_n}}$ we have found is not equal to 1, but is relatively very close, so we can say that we have some errors in calculating the \textbf{drag force versus temperature}.}
        
        {\textbf{Percentage uncertainty} in measurement of \textbf{drag force versus temperature} is $\left|\frac{1-\overline{F_{D_n}}}{\overline{F_{D_n}}}\right|\cdot{100\%} = \textbf{5.112\%} \approx \textbf{5.12\%}$}
        



