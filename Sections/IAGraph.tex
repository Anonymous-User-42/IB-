
\subimport{./}{FDGraph1}            
%\subimport{./}{FDGraph}
            
\textit{Upon close visual observation, we see that the difference in the plotted values of that of the \textbf{simulation} and \textbf{experimental values} of the \textbf{drag force} versus \textbf{temperature} from the \textbf{F-T} graph considering all data points are very \textbf{minute} to the extent that it would be right to say and consider that the experimental values are both \textbf{accurate} and \textbf{precise} in relation to that of the \textbf{literature/theoretical/simulation values}.}

\textit{In order to model the system with accuracy, a \textbf{quadratic regression model} was utilized to model the \textbf{line of best-fit}.}
        
\textit{It is evident from studying the system \textbf{numerically} and \textbf{graphically} that the system exhibits an \textbf{logarithmic decay} with time for the parameter that is being investigated. This points out to \textbf{one conclusion}, that the \textbf{parameter of the system undergoes logarithmic decay not linear decay}, according to their various \textbf{inert energies}.}

\textit{\textbf{Note}: Due to constraints of resources, a \textbf{quadratic regression model} was utilized instead of a \textbf{logarithmic regression model}, but the part of the model that we are to analyze exhibits logarithmic decay over temperature and is \textbf{accurately similar} to a quadratic regression model when restricted in the domain from 10 $^\circ$ C to 90 $^\circ$ C.}
	 
\textit{This \textbf{contradicts} the \textbf{initial hypothesis} laid out prior to beginning the investigation as if, the system had to model \textbf{similarly} to the inverse exponential, then the logarithmic behavior that we observe would have not existed, but we see that this is not the case.}        
        

\textit{It would be right to say that the \textbf{initial hypothesis} that was laid out prior, beginning the experimentation was \textbf{incorrect} and is \textbf{not} a valid statement.}        
    
    

