
\subimport{./}{FDGraph1}            
%\subimport{./}{FDGraph}
            
{Upon close visual observation, we see that the difference in the plotted values of that of the \textbf{simulation} and \textbf{experimental values} of the \textbf{drag force} versus \textbf{temperature} from the \textbf{F-T} graph considering all data points are very \textbf{minute} to the extent that it would be right to say and consider that the experimental values are both \textbf{accurate} and \textbf{precise} in relation to that of the \textbf{literature/theoretical/simulation values}.}

{In order to model the system with accuracy, a \textbf{quadratic regression model} was utilized to model the \textbf{line of best-fit}.}
        
{The system exhibits a \textbf{logarithmic decay} with time with respect to the parameter that is being researched, as evidenced by the \textbf{numerical} and \textbf{graphical} analysis of the system. This leads to \textbf{one conclusion}, that depending on their various \textbf{inert energies}, the \textbf{parameter of the system exhibit logarithmic decay rather than linear decay}.}

{\textbf{Note}: A \textbf{quadratic regression model} has been used rather than a \textbf{logarithmic regression model} due to resource constraints, although the phase of the model we'll be examining exhibits logarithmic decay with respect to temperature and is \textbf{quite identical} with the quadratic regression model when limited to the 10 $^\circ$ C to 90 $^\circ$ C range.}
	 
{These findings \textbf{contradict} the \textbf{initial hypothesis}, which said that if the system had to model in accordance with the \textbf{inverse exponential function}, the \textbf{logarithmic phenomenon we have found wouldn't have existed}.}        
        

{It would be right to say that the \textbf{initial hypothesis} that was laid out prior, beginning the experimentation was \textbf{incorrect} and is \textbf{not} a valid statement.}        
    
    

