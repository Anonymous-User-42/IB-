\begin{table}[H]
    \centering
        \begin{tabular}{|c|c|}
        \hline
        \hline
        {Physical Quantity} & {Symbol} \\
        \hline
        \hline
        {Flow velocity} & \textit{v} \\
        \hline
        {Drag coefficient} & \textit{$C_D$} \\
		\hline        
        {Temperature} & \textit{T} \\
        \hline
        {Radius of spherical mass} & \textit{r} \\
        \hline
        \hline
        \end{tabular}
    \caption{\textit{General physical quantities employed in this investigation}}
\end{table}

{\textbf{Note}: In theory, the radius of the spherical mass and the flow velocity  incorporated in research could of any arbitrary value. For the purpose of this investigation, we shall specifically use masses of radius $2.5\times10^{-2}$ m and the flow velocity shall be set constant at 0.1 m/s.}

\begin{table}[H]
    \centering
        \begin{tabular}{|c|c|c|}
        \hline
        \hline
        {Independent Variable} & {Dependent Variable} & {Controlled Variable} \\
        \hline
        \hline
        {Temperature} & {Drag Force} & {Fluid medium} \\
        \hline
        {-} & {-} & {Radius of the spherical mass} \\
        \hline
        {-} & {-} & {Flow velocity} \\
        \hline
        \hline
        \end{tabular}
    \caption{{Segregation of employed variables as IV, DV or CV}}
\end{table}

{\textbf{Note}: The flow velocity, type of fluid medium and the radius of the mass is no longer variable as we have defined a set value to it.}

