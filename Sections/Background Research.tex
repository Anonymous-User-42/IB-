{Before we begin our investigation, we must first understand some important facts, equations, and laws, as well as be familiar with the ideas involved.}
        
\subsection{{Air Drag/Fluid Resistance}}
        
    {The force working in opposition to a moving object in a fluid medium is known as air drag or fluid resistance. The drag force is proportional to the \textbf{velocity squared}. This is due to the fact that we will be dealing and working with relatively high speeds, as evidenced by the small Reynold's number.}
            
    {Drag forces gradually reduce the fluid velocity relative to the solid mass in the fluid's path.}
            
    {The general Drag equation is mathematically defined as,}
            
        $$F_D = \frac{1}{2}\rho v^2C_DA$$
           
    {Where $F_D$ is the \textbf{Air/Fluid resistance} between the mass and the fluid, $\rho$ is the \textbf{fluid density}, \textit{v} is the \textbf{object speed} relative to the fluid, $C_D$ is \textbf{velocity decay constant} (damping constant) and A is the \textbf{area of cross section}.}
            
    {\textbf{Note}: The \textbf{velocity decay constant}, $C_D$ for the particular case that we are investigating, that is on \textbf{spherical bodies} has a set defined value of \textbf{0.47}.}
            
\subsection{{Temperature dependence of thermal expansion \\ coefficient}}

	{The thermal expansion coefficient is defined as,}
	
		$$\alpha_{L} = \frac{1}{L}\cdot\frac{\partial L}{\partial T}$$
	
	{Where, L is the \textbf{length measurement}, $\alpha_{L}$ is the \textbf{thermal expansion coefficient} in the dimension of the length measurement and T is the \textbf{temperature}.}	
	
	{Because, the length measurement that we are dealing with is Volume, the above equation reduces to,}

		$$\alpha_{V} = \frac{1}{V}\cdot\frac{dV}{dT}$$
	
	{This clearly indicates that $\alpha_{V}$ or the cubic expansion coefficient is a function dependent of temperature.}	
	
	{The below table encompasses the values of the cubic expansion coefficient of water (“Volumetric (Cubic) Thermal Expansion”) at certain temperatures.}	
	
	\begin{table}[H]
		\centering
		\begin{tabular}{|c|c|}
		\hline
		\hline
		{Temperature $^\circ C$} & {Cubic thermal expansion coefficient $1/^\circ C$} \\
		\hline
		\hline
		0 & -0.000050 \\
		\hline		
		10 & 0.000088 \\
		\hline
		20 & 0.000207 \\
		\hline
		30 & 0.000303 \\
		\hline
		40 & 0.000385 \\
		\hline
		50 & 0.000457 \\
		\hline
		60 & 0.000522 \\
		\hline
		70 & 0.000582 \\
		\hline
		80 & 0.000640 \\
		\hline
		90 & 0.000695 \\
		\hline
		\hline 
		\end{tabular}
	
	\end{table}

\subsection{{Temperature dependence of density}}
            
	{We must know the fundamental relation between change in \textbf{temperature} on change in \textbf{density}.}
            
    {We know that,}
    
    		$$\rho = \frac{m}{V}$$
            
	{Where, $\rho$ is the \textbf{density} of a particular substance, m is its \textbf{mass} and V is its \textbf{volume}.}            
          
	{Therefore, we have}          
            
		$$\rho \propto \frac{1}{V}$$            
            
	{Therefore, we infer that, \textbf{density} is \textbf{inversely proportional} to \textbf{volume} of the substance, here the \textbf{fluid}.}            
            
	{We have a equation for the temperature dependence on density from (“Liquids - Densities vs. Pressure and Temperature Change”). That is,}            
            
		$$\rho = \frac{\rho_{0}}{1 + \gamma\cdot\Delta T}$$            

	{Where $\rho$ is the \textbf{current density} of a particular substance, $\rho_{0}$ is the \textbf{initial density} of a particular substance, $\gamma$ is the \textbf{volumetric/cubic thermal expansion coefficient} and $\Delta T$ is the change in the temperature from the initial state.}            

	{As $\left(1 + \gamma\cdot\Delta T\right)^{-1}$ is of the form $\left(1 + x\right)^{-1}$ we have,}
	
		$$\left(1 + \gamma\cdot\Delta T\right)^{-1} = 1 - \left(\gamma\cdot\Delta T\right) + \left(\gamma\cdot\Delta T\right)^2 - \left(\gamma\cdot\Delta T\right)^3 \cdots$$

	{If we ignore the higher order terms of $\left(\gamma\cdot\Delta T\right)$, as they are negligibly small, we have,}
            
		$$\rho = \rho_{0}\left(1 - \gamma\cdot\Delta T\right)$$            

	{Because $\gamma$ is a function of T, $\gamma = \gamma_{T}$, therefore we have,}

		$$\rho = \rho_{0}\left(1 - \gamma_{T}\cdot\Delta T\right)$$
                      
\subsection{{Computer Simulation Software}}
        
    {To model the \textbf{drag force} on a \textbf{sphere in fluid flow}, we will need the help of computer technology. Softwares such as MATLAB and Mathematica would compute and yield solutions for necessary simulations.}

    {This study uses a \textbf{MATLAB} script to model and simulates the system.}
        
        


