
\subsection{\textit{Drag Equation for the Study}}
            
   \textit{We know that,}
            
		$$F_D = \frac{1}{2}\rho v^2C_DA$$

	\textit{Also,}
		
		$$\rho = \rho_{0}\left(1 - \gamma\cdot\Delta T\right)$$

	\textit{Substituting the second equation in the first equation we have,}
	
		$$F_D = \frac{1}{2}\rho_{0}v^2C_DA\cdot\left(1 - \gamma\cdot\Delta T\right)$$

	\textit{This equation can be rewritten as,}

		$$F_D = \frac{1}{2}\rho_{0}v^2C_DA\cdot\left(1 - \gamma\cdot\left(T - T_{0}\right)\right)$$

	\textit{When further reducing the variables to constants the following equation reduces to,}

		$$F_D = 4.61\cdot 10^{-3}\cdot\left(1 - \gamma\cdot T\right)$$

	\textit{\textbf{Note}: The initial density and temperature taken into account is 999.83 $kg/m^3$ and ${0}^\circ C$ or 271.16 K.}

	\textit{\textbf{Note}: This equation is derived considering the Celsius unit for temperature difference as the least count for both the units (Celsius and Kelvin) are the same.}

\subsection{\textit{Equation used for further Investigation}}

	\textit{The \textbf{equation used for further investigation} we shall be using in this investigation is:}
            
   	\begin{equation}
      	F_D = 4.61\cdot 10^{-3}\cdot\left(1 - \gamma_{T}\cdot T\right)
      	\label{eq1}
    	\end{equation}
            
	\textit{Where $\gamma_{T}$ is the cubic thermal expansion coefficient of water and T is the magnitude of the temperature of water.}
            
%	\textit{\textbf{Note}: All the equations mentioned above shall be inputted algorithmically in an computer program/software and shall be numerically estimated to about an accuracy of approximately $\pm10^{-8}$ of each time step interval.}
            
            




